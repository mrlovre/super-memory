\documentclass[12pt, a4paper]{article}

\usepackage[utf8]{inputenc}
\usepackage[T1]{fontenc}
\usepackage[pdftex]{graphicx}
\usepackage{booktabs}
\usepackage{amsmath}
\usepackage{amssymb}
\usepackage[left=2.5cm,right=2.5cm,top=3cm,bottom=3.5cm]{geometry}
\usepackage{indentfirst}
\usepackage[activate={true,nocompatibility},final,tracking=true,kerning=true,spacing=true,factor=1100,stretch=10,shrink=10]{microtype}
\usepackage{float}

\renewcommand{\figurename}{Slika}

\microtypecontext{spacing=nonfrench}

\allowdisplaybreaks

\title{Neizrazito, evolucijsko i neuroračunarstvo:\\
  Izvješće uz 7. laboratorijsku vježbu -- \\Klasifikacija umjetnom neuronskom mrežom treniranom genetskim algoritmom}
\author{Lovre Mrčela}

\date{13. siječnja 2017.}

\begin{document}

\maketitle

\paragraph{Zadatak 1.}
Razmotrite jedan neuron koji ima samo jedan ulaz.
Njegov izlaz tada će biti određen izrazom:

$$ y= \dfrac{1}{1 + \dfrac{|x - w|}{|s|}}$$
Pretpostavite da je u neuron pohranjena vrijednost $w=2$. Nacrtajte \textit{na istom grafu} ovisnost $y(x; w=2)$ za tri slučaja: za $s=1$, za $s=0.25$, te za $s=4$ (svaku različitom bojom ili stilom linije). Za raspon apscise uzmite interval $[-8, 10]$. Razumijete li sada kako $s$ utječe na izlaz neurona $y$? Kako će izgledati izlaz neurona koji ima dva ulaza i što se tada kontrolira parametrima $s_1$ i $s_2$?

\end{document}
